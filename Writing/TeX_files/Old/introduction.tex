\chapter{Introduction}
Political revolutions are almost prototypical complex systems. They show self-organisation, bifurcations, chaos and order. Whilst the field of complex systems has developed dramatically in recent years, there has not been an update to models of revolutions. Most models\cite{1}\cite{2}\cite{3} are based on a model developed in a short 2002 article by Epstein\cite{epstein}, with added nuance. Whilst an important paper, it is just one way of moelling hte behaviour. Morever, this model was built well in advance of the post-digital era, emphasising the role of spatial processes such as arrets in the street. In doing so it sturggles to account for revolutions such as the Arab Spring which relied on fundamentally different processes such as social networks and remote organisation. Note that whilst set-up in the model is potentially metaphorical, the dynamics inherent in it commit it to something like this. This paper will attempt to to develop two different routes to modelling political revolutions as alternatives. The first model follows an compartmental model inspired by mathematical epidimiology. This is inspired by the similarities between the spread of disease and the spread of revolutionary ideas. The second model is a network model like Epstein's but instead of having space as the fundamental element, takes a more general perspective, emphasising the transmission of ideas. This allows the model to account for idea spread without commiting to the means of transmission, allowing it to better account for more recent revoutions such as the Arab Spring as well as still offering an analysis of pre-digital revolutions.

\section{What has come before}
\subsection{What is a revolution and some examples}
What we are interested in: non-violent, 'bottom-up' movements where a significant fraction of citizens peacefully overthrow a corrupt governing authority.\\
What we are not interested in: violent protest or coups. Both involve very different parameters rather than the communication of ideas and the latter does not act like a complex system.

\subsection{Political theories of revolution}
Maybe some marxist thought.\\
In on Tyranny, Timothy Snyder recognises that, for resistance to succeed, “ideas about change must engage people of various backgrounds who do not agree about everything”.\\
Effect of internet making spatial interactions less important. (Recognise that they are important in the final stages of a revolution.)

\section*{Political theories of revolution}
Maybe some marxist thought.\\
In 'On Tyranny', Timothy Snyder recognises that, for resistance to succeed, “ideas about change must engage people of various backgrounds who do not agree about everything”.\\
Effect of internet making spatial interactions less important. (Recognise that they are important in the final stages of a revolution.)

